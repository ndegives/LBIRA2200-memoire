\section{Conclusion}

Ce travail s'inscrit dans un grand travail de recherche sur les flux hydriques au sein des plantes.
L'intérêt porté à ces flux n'est pas nouveau, mais il reste beaucoup de zone à éclaircir.
Certains processus ont pas encore été suffisamment explorés.
Ces flux sont d'autant plus important aujourd'hui au vu du tournant qu'a commencé à prendre la sélection végétale pour inclure la résilience des plantes pour considérer une culture.
\newline

La recherche sur les racines a connu un intérêt plus tardif que pour les parties aériennes de par la difficulté que représente l'échantillonnage de celle-ci.
Beaucoup de processus majeurs sont désormais bien compris et décrit, mais le système racinaire et son influence sur les flux hydriques manque encore quelque peu de précision.
Le système racinaire nécessite, non seulement de prendre en compte beaucoup d'aspect, mais également d'articuler tous ces éléments ensemble pour parvenir à un résultat final le plus précis possible.
\newline

Le sorgho, longtemps cultivés dans des régions semi-arides, pourrait s'imposer dans de nouvelles régions suite aux dérèglements climatiques.
Cette céréale possède de nombreux atouts face aux maïs, notamment des besoins hydriques sensiblement plus faibles, et offre également une multitude de débouchés.
Cette importante résilience hydrique représente un intérêt majeurs face au dérèglement climatique et la question se pose de la contribution du système racinaire dans cette tolérance à la sécheresse.
\newline

Le CIPF a, suite aux nombreux avantages qu'a le sorgho, commencé à prêter attention à cette céréale pour identifier et partager les nombreuses opportunités qu'elle peut apporter dans les régions comme la Belgique.
Des vitrines de tests de cultures de différentes variétés de sorgho sont ainsi réalisés en Belgique par le CIPF depuis 2013.
Les systèmes racinaires échantillonnés lors de ce travail proviennent de six variétés de Sorgho fourrager testées depuis au moins trois ans au CIPF : Amiggo; RGT Biggben; ES Hyperion; KWS Juno ; RGT Swingg et Vegga.
Afin de pouvoir observer au maximum les systèmes racinaires de ces variétés, des premiers échantillons racinaires ont été récupérés en champs après récolte et d'autres ont été réalisés en rhizotron pour en observer la croissance.
\newline

La première étape a été de quantifier des caractéristiques racinaires chez les six variétés de sorgho.
Cinq paramètres racinaires (Dmin, Dmax, Drange, DIDm et IBD) identifier par \cite{pages_seeking_2018} comme capturant assez bien l'entièreté d'un système racinaire ont été retenus pour caractériser les variétés de sorgho observées.
Ces paramètres ont donc été quantifier et les différences ou similitude entre variétés ont été identifiées à l'aide de tests statistiques.
Il en est sorti que les six variétés était, vis-à-vis de ces cinq paramètres, plutôt similaires.
Seuls le Dmin de Vegga et les six DIDm se sont avérés être statistiquement différents parmi les variétés.
\newline

Le modèle utilisé dans ce travail pour modéliser l'architecture des systèmes racinaires est ArchiSimple.
Ce modèle fonctionne sur base de cinq principaux processus : émission, élongation, ramification, croissance radiale et abscission.
Pour fonctionner, ces processus utilise une vingtaine de paramètres demandés en input du modèle.
Le faible nombre d'inputs est un atout de ArchiSimple et est permis par le fait que l'intensité des processus est souvent basé sur les diamètres racinaires.
Cette approche utilisant les diamètres est exclusive à ArchiSimple et assez intéressante dès lors que toutes les relations impliquant le diamètre dans le modèle ont été prouvés comme évoluant avec le diamètre
\newline

Les systèmes racinaires obtenus via ArchiSimple étaient assez semblables pour les six variétés.
Ce résultat est cohérent au vu des faibles variations observées parmi les paramètres fournit en input du modèle.
L'architecture racinaire du maïs a également été générée à l'aide du modèle afin de pouvoir comparer les résultats obtenus pour le sorgho à ceux du maïs.
Il est alors apparu que les architectures générées pour le sorgho occupaient un plus grand volume de sol de par la longueur des racines latérales et les directions, moins verticales, dans lesquelles les racines adventives se développaient.
Cela va dans le sens de nombreux articles indiquant que le sorgho jouit d'un système racinaire puissant qui couvrent plus de volume de sol que le maïs avec des racines pouvant atteindre jusqu'à deux mètres de profondeur.
\newline

Il est apparu que, bien que les systèmes racinaires générés n'étaient pas démesurés, ArchiSimple manque de réalisme dans sa construction de système racinaire adventif.
Les racines adventives, toutes émises avec le même diamètre, limitent la justesse que pourrait avoir ArchiSimple dans la modélisation de certains systèmes racinaires et particulièrement ceux du sorgho.
Les racines adventives devraient observer une augmentation de diamètres au fur et à mesure de leurs apparitions et cela n'est pas pris en compte par ArchiSimple.
Les diamètres étant au centre de beaucoup de dynamiques dans le modèle, cela engendre différents problèmes dans les systèmes racinaires obtenus.
Le paramètre EL, liant le diamètre à l'élongation racinaire, présente également le désavantage d'être constant alors qu'il a été montré que l'élongation racinaire n'est pas linéaire avec le diamètre et est sujette à une diminution avec le temps et la profondeur.
ArchiSimple est donc, bien que très prometteur pour la modélisation d'architecture de système racinaire, parfois un peu lacunaire et mériterait certaines améliorations.
\newline

Finalement, le sorgho ne présente pas encore le profil d'un candidat aux grandes cultures belges.
Les printemps peuvent encore être déficitaire en température pour cette culture et les rendements actuels égalent difficilement ceux de son homologue le maïs.
Néanmoins, il convient de continuer à s'y intéresser tant comme sujet de recherche que comme alternative de culture pour le futur.
Le sorgho représente aujourd'hui un exemple de résilience hydrique et identifier toutes ses particularités peut mener à l'amélioration d'autres cultures en leur inculquant les traits d'intérêts à la résistance aux stress hydriques.
La sélection de sorgho est également encore sujette à de nombreuses avancées en termes de rendements puisque celle-ci est loin d'être aboutie pour nos régions.
Le sorgho pourrait alors bien, à terme, venir à égaler les rendements du maïs tout en conservant une meilleure résilience, ce qui rendrait cette culture incontournable dans de plus en plus de régions du monde.