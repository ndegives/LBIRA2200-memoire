\section{Objectifs}

Le sorgho, depuis longtemps cultivé dans les régions tropicales semi-arides d'Afrique et d'Asie, est une culture que l'on pourrait voir s'étendre dans des régions plus au nord.
Cet intérêt particulier qui lui est voué provient du fait que le sorgho offre les mêmes débouchés que le maïs qui figure parmi les plus grandes cultures dans le monde.
Il serait imaginable que le sorgho, après avoir bénéficié de recherche et sélection, monte au classement des céréales les plus cultivées dont il occupe déjà la cinquième place.
\newline

Le sorgho, en plus d'offrir une multitude de services, possède un atout qui intéresse le monde agricole actuellement : son impressionnante résistante à la sécheresse.
Plusieurs mécanismes morphologiques et physiologiques participant à cette résilience hydrique ont déjà été identifiés.
Toutefois, plusieurs pistes situées dans le système racinaire restent à ce jour inexplorées.
Il est connu et prouvé que le sorgho possède un système racinaire assez profond et étendu grâce à ses nombreuses racines nodales.
Cependant, le développement de l'architecture racinaire ainsi que l'impact qu'a celle-ci sur l'extraction d'eau du sol sont à ce jour peu étudié.
\newline

Ce mémoire s'inscrit dans cette démarche et s'intéresse donc à la modélisation du système racinaire de sorgho afin d'en identifier les traits particuliers.
Les objectifs sont alors :
\begin{enumerate}
    \item Caractériser et quantifier le système racinaire de six variétés de sorgho testées au CIPF. 
    \item Identifier les potentielles différences entre ces variétés.
    \item Modéliser les architectures des systèmes racinaires de ces variétés.
    \item Comparer les modélisations de systèmes racinaires de sorgho à ceux du maïs (similaire mais moins tolérant au stress hydrique).
\end{enumerate}

Plus globalement, ce travail s'inscrit dans une étude qui vise à comprendre les flux hydriques au sein des plantes.
Dans l'optique, nécessaire à ce jour, de sélection variétale et génétique de cultures plus résilientes aux fréquentes sécheresses, le sorgho a aujourd'hui une avance confortable.
Comprendre comment le système racinaire du sorgho participe à lui offrir une meilleure tolérance aux stress hydriques que le maïs offre des pistes pour la sélection des variétés d'une multitude d'autres espèces face aux changements climatiques.
Néanmoins, ce travail est loin d'être exhaustif et ne peut, à son échelle, qu'apporter une pierre à l'édifice du travail de recherche bien large sur les flux hydriques au sein des plantes. 
Comme mentionné précédemment, la modélisation de système racinaire seule n'a que peu de valeur.
C'est en incluant cela dans un réseau (comme présentée en figure \ref{fig:modelling}) qu'il est alors possible d'améliorer la compréhension globale des flux hydriques au sein du système sol-plante.