\section*{Introduction}
\addcontentsline{toc}{section}{Introduction}

Depuis quelques années, l’agriculture a dû ajouter à la liste de ses défis les dérèglements climatiques.
En effet, les variations de température, la fréquence des événements météorologiques extrêmes tels que les sécheresses, les inondations et les tempêtes, ainsi que les changements dans les schémas de précipitations, ont des effets directs sur la production alimentaire mondiale.
Les effets du climat se font déjà sentir dans de nombreuses régions du monde, entraînant des pertes de récoltes, une diminution de la qualité des aliments, une augmentation des maladies des cultures et une insécurité alimentaire croissante.
L’agriculture se doit alors de s’adapter aux dérèglements climatiques et trouver des moyens innovants de produire tout en préservant l’environnement. 
\newline

Dans ce contexte de pressions croissantes sur les ressources naturelles, la recherche agricole doit se diriger vers des pratiques plus durables qui permettent une utilisation plus efficace des ressources, une réduction des émissions de gaz à effet de serre, une amélioration de la qualité des sols et de la biodiversité, et une plus grande résilience face aux chocs climatiques \citep{oecd_building_2021}.
Cette transition vers une agriculture durable et résiliente peut entre autres se réaliser par la sélection variétale ou de culture.
La recherche du rendement doit désormais être mise en parallèle avec d'autres traits devenus importants.
Ce concept a déjà été formalisé par \cite{donald_breeding_1968} qui définit un "idéotype de culture" comme la variété la mieux adaptée à une situation de production donnée.
Les objectifs étant de plus en plus diversifiés suite à l'incertitude climatique, il est plus que jamais nécessaire de penser "idéotype" et non plus exclusivement "production".
\newline

Une bonne compréhension des flux hydriques au sein du système sol-plante est essentielle pour améliorer la résilience des cultures. Ces flux dépendent de nombreux processus qu'il est nécessaire de décrire, quantifier et mettre en relation les uns avec les autres \citep{lobet_plant_2014}. 
Il est donc encore complexe de parvenir à quantifier précisément ces flux pour différentes espèces. Les processus ayant lieu dans les parties racinaires sont d'autant plus complexes à appréhender du fait qu'ils sont difficiles à observer. 
\newline
 
Le sorgho (Sorghum bicolor) est une céréale plutôt tolérante à la sécheresse.
Plusieurs caractéristiques participant à cette résistance face au stress hydrique ont déjà été identifiées.
Toutefois, la caractérisation du système racinaire du sorgho reste à ce jour assez superficielle.
Ce travail découle de ce constat et cherche donc à identifier les caractéristiques racinaires propres au sorgho.
Il vise en premier lieu à quantifier ces caractéristiques pour ensuite permettre la modélisation d'un système racinaire de sorgho et ainsi participer à une meilleure compréhension de cette résilience hydrique que possède le sorgho.